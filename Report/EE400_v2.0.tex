%%%%%%%%%%%%%%%%%%%%%%%%%%%%%%%%%%%%%%%%%
% University Assignment Title Page 
% LaTeX Template
% Version 2.0 (21/04/18)
% Modified by
% Erdem TUNA &
% Halil TEMURTAŞ
%
% This template has been downloaded from:
% http://www.LaTeXTemplates.com
%
% Original author:

% Instructions for using this template:
% This title page is capable of being compiled as is. This is not useful for 
% including it in another document. To do this, you have two options: 
%
% 1) Copy/paste everything between \begin{document} and \end{document} 
% starting at \begin{titlepage} and paste this into another LaTeX file where you 
% want your title page.
% OR
% 2) Remove everything outside the \begin{titlepage} and \end{titlepage} and 
% move this file to the same directory as the LaTeX file you wish to add it to. 
% Then add \input{./title_page_1.tex} to your LaTeX file where you want your
% title page.
%
%%%%%%%%%%%%%%%%%%%%%%%%%%%%%%%%%%%%%%%%%
%\title{Title page with logo}
%----------------------------------------------------------------------------------------
%	PACKAGES AND OTHER DOCUMENT CONFIGURATIONS
%----------------------------------------------------------------------------------------
\documentclass[a4paper,12pt]{article}
\usepackage[a4paper, total={5.8in, 7.6in}]{geometry}
%\documentclass[12pt]{article}
\usepackage[english]{babel}
\usepackage[utf8x]{inputenc}
\usepackage{amsmath}
\usepackage{graphicx}
\usepackage[colorinlistoftodos]{todonotes}
\usepackage{gensymb} % this could be problem
\usepackage{float}
\usepackage{fancyref}
\usepackage{subcaption}
\usepackage[toc,page]{appendix} %appendix package
\usepackage{xcolor}
\usepackage{listings}
\usepackage{xspace}


\usepackage{amssymb}
\usepackage{nicefrac}
\usepackage{gensymb}
\usepackage{xspace}
\usepackage{fancyhdr}


\usepackage[final]{pdfpages}


\usepackage{array} %allows more options in tables
\usepackage{pgfplots,pgf,tikz} %coding plots in latex
\usepackage{capt-of} % allows caption outside the figure environment
\usepackage[export]{adjustbox} %more options for adjusting the images
\usepackage{multicol,multirow,slashbox} % allows tables like table1
%\usepackage[hyperfootnotes=false]{hyperref} % clickable references
\usepackage{epstopdf} % useful when matlab is involved
\usepackage{placeins} % prevents the text after figure to go above figure with \FloatBarrier 
%\usepackage{listingsutf8,mcode} %import .m or any other code file mcode is for matlab highlighting

\setcounter{secnumdepth}{6} %depth of structures in table of contents
\setcounter{tocdepth}{6} %depth of structures in table of contents


\newcommand\nd{\textsuperscript{nd}\xspace}
\newcommand\rd{\textsuperscript{rd}\xspace}
\newcommand\nth{\textsuperscript{th}\xspace} %\th is taken already


\definecolor{mGreen}{rgb}{0,0.6,0} % for python
\definecolor{mGray}{rgb}{0.5,0.5,0.5}
\definecolor{mPurple}{rgb}{0.58,0,0.82}
\definecolor{mygreen}{RGB}{28,172,0} % color values Red, Green, Blue for matlab
\definecolor{mylilas}{RGB}{170,55,241}


\newcommand{\specialcell}[2][c]{%
  \begin{tabular}[#1]{@{}c@{}}#2\end{tabular}}


\lstdefinestyle{CStyle}{
    commentstyle=\color{mGreen},
    keywordstyle=\color{magenta},
    numberstyle=\tiny\color{mGray},
    stringstyle=\color{mPurple},
    basicstyle=\footnotesize,
    breakatwhitespace=false,         
    breaklines=true,
    frame=single,
    rulecolor=\color{black!40},                 
    captionpos=b,                    
    keepspaces=true,                 
    numbers=left,                    
    numbersep=5pt,                  
    showspaces=false,                
    showstringspaces=false,
    showtabs=false,                  
    tabsize=2,
    language=C
}


\lstset{language=Matlab,%
    %basicstyle=\color{red},
    breaklines=true,%
    frame=single,
    rulecolor=\color{black!40},
    morekeywords={matlab2tikz},
    keywordstyle=\color{blue},%
    morekeywords=[2]{1}, keywordstyle=[2]{\color{black}},
    identifierstyle=\color{black},%
    stringstyle=\color{mylilas},
    commentstyle=\color{mygreen},%
    showstringspaces=false,%without this there will be a symbol in the places where there is a space
    numbers=left,%
    numberstyle={\tiny \color{black}},% size of the numbers
    numbersep=9pt, % this defines how far the numbers are from the text
    emph=[1]{for,end,break},emphstyle=[1]\color{red}, %some words to emphasise
    %emph=[2]{word1,word2}, emphstyle=[2]{style},    
}



\makeatletter
\renewcommand\paragraph{\@startsection{paragraph}{4}{\z@}%
            {-2.5ex\@plus -1ex \@minus -.25ex}%
            {1.25ex \@plus .25ex}%
            {\normalfont\normalsize\bfseries}}
\makeatother
\setcounter{secnumdepth}{5} % how many sectioning levels to assign numbers to
\setcounter{tocdepth}{6}    % how many sectioning levels to show in ToC



% For Blank Page -------------
\newcommand{\blankpage}{
	\- \\[8.5cm]	
	{ \centering This Page Intentionally Left Blank \par }
	\- \\[8.5cm]
}
% ---------------------------

%\begin{figure}[H]
%	\setlength{\unitlength}{\textwidth} 
%	\centering
%	\begin{subfigure}{.5\textwidth}
%  		\centering
%  		\includegraphics[width=0.48\unitlength]{SubFigure1}
%  		\caption{\label{fig:As12}SubFigure2 }
%	\end{subfigure}%
%	\begin{subfigure}{.5\textwidth}
%  		\centering
%		\includegraphics[width=0.48\unitlength]{SubFigure2}
%  		\caption{\label{fig:As12}SubFigure2 }
%	\end{subfigure}
%\caption{\label{fig:As12}Figure }
%\end{figure}



\begin{document}

\begin{titlepage}

\newcommand{\HRule}{\rule{\linewidth}{0.5mm}} % Defines a new command for the horizontal lines, change thickness here

\center % Center everything on the page
%----------------------------------------------------------------------------------------
%	LOGO SECTION
%----------------------------------------------------------------------------------------

\includegraphics[scale=0.3]{odtuee.png}\\[1cm]
% Include a department/university logo - this will require the graphicx package
 
%----------------------------------------------------------------------------------------

 
%----------------------------------------------------------------------------------------
%	HEADING SECTIONS
%----------------------------------------------------------------------------------------

\textsc{\LARGE Middle East Technical University}\\[1.5cm] % Name of your university/college
\textsc{\Large Department of Electrical and Electronics Engineering }\\[0.5cm] % Major heading such as course name
 % Minor heading such as course title

%----------------------------------------------------------------------------------------
%	TITLE SECTION
%----------------------------------------------------------------------------------------

\HRule \\[0.4cm]

{ \huge \bfseries \large EE400 Summer Practice II \\ Report}\\[0cm] % Title of your document
\HRule \\[1cm]
 
%----------------------------------------------------------------------------------------
%	AUTHOR SECTION
%----------------------------------------------------------------------------------------

\begin{minipage}{0.38\textwidth}
\begin{flushleft} \large
	\textbf{Student Name:} \\
		\textit{Halil Temurtaş} \\
	\textbf{Student ID:} \\ 
		\textit{2094522} \\
	\textbf{SP Beginning Date:} \\
		\textit{09.07.2018} \\
	\textbf{SP End Date:}\\
		\textit{03.08.2018} \\
	\textbf{EE300 SP Location:} \\ 
		\textit{TÜRKSAT A.Ş.} 
\end{flushleft}
\end{minipage}
\begin{minipage}{0.6\textwidth}
\begin{flushright} \large
	\textbf{EE400 SP Company Name:} \\ 
		\textit{ASELSAN A.Ş.} \\
	\textbf{EE400 SP Company Division:} \\ 
		\textit{HBT Test ve Süreç Tasarımı Mdl.} \\
	\textbf{Supervisor Engineer:} \\
		\textit{Pınar Kırıkkanat} \\
	\textbf{SE Contact Mail:} \\
		\textit{pkirikkanat@aselsan.com.tr}  \\
	\textbf{SE Contact Phone:} \\
		\textit{+90 (312) 592 19 17 } 
\end{flushright}
\end{minipage}\\[0.8cm]

% If you don't want a supervisor, uncomment the two lines below and remove the section above
%\Large \emph{Author:}\\
%John \textsc{Smith}\\[3cm] % Your name

%----------------------------------------------------------------------------------------
%	DATE SECTION
%----------------------------------------------------------------------------------------

{\large 21/08/2018}\\[1cm] % Date, change the \today to a set date if you want to be precise


\vfill % Fill the rest of the page with whitespace

\end{titlepage}


\blankpage



\tableofcontents
\newpage


%\begin{abstract}
%Your abstract.
%\end{abstract}

\section{Introduction}
\-\indent 
	I performed my summer practice in ASELSAN A.Ş., one of leading defence industry companies in Turkey. my internship lasted 20 days and Pınar Kırıkkanat, an electronics engineer in ASELSAN was my supervisor and assisted me in my summer practice.

	The summer practice started with an orientation program that briefly explains the company and how the works are handled. Following that, mandatory educations like occupational safety and health education is given to the interns by the company. After the educations, inters were sent to their assigned departments and divisions. I was similarly sent to the HBT Division to perform my summer practice.
	
	In the first half of my internship, I was given time to observe, learn and participate the mechanical and electrical test conducted at our division. Mainly on ASELSAN 9661 Series Radios, I mostly observed and participated on the electrical and environmental tests of the equipments produced at the Communication \& Information Technologies Vice Presidency, known as HBT. .
	
	In the second part of my internship, I was given time to observe the work done behind the testing, in other words process design and management. In this part of my internship, I participated on documentation and research activities for the ULAK Base station of the ASELSAN. Since the work done at this stage can be mostly considered as classified information, I will mention the basics of what I have done in this part.
	
	In this, report, I will start with an introduction, that covers what was done in my summer practice. Then, I will continue with a company description section in which the general description about ASELSAN is given. After that part, the work done in my summer practice will be explained. Lastly, I will finishe the report with an conclusion part. 
	

\- \\[3cm]
 

\section{Description of the Company}
\- \indent
	In this chapter, I will introduce the company in five main parts:



\subsection{Company Name}
\-
\indent ASELSAN A.Ş. or \textbf{AS}KERİ \textbf{EL}EKTRONİK \textbf{SAN}AYİ A.Ş.


\subsection{Company Locations}
\-\indent
	ASELSAN has six campuses in Turkey, one of them being in Istanbul, other campuses are located at Ankara. Throughout my summer practice, I spent my time at \textbf{Macunköy} Facilities . 
\\
\\
\textbf{ Address / Macunköy Facilities:} Mehmet Akif Ersoy Mahallesi 296. Cadde No: 16, 06370 Yenimahalle-Ankara, Türkiye
\\
\\
\textbf{ Phone:} +90 (312) 592 10 00
\\
\\
\textbf{ Fax:} +90 (312) 354 13 02 / +90 (312) 354 26 69


\subsubsection{Macunköy Facilities}
\- \indent

	Macunköy Facilities was established on an area of total 186.000 m2, 110.000 m2 of which is the closed area. \textbf{General Directorate}, \textbf{SST} Group, \textbf{UGES} Group and parts of \textbf{HBT} and \textbf{REHİS} Groups are at Aselsan Macunköy Facilities.



\subsubsection{Other Facilities}


\begin{itemize}
\item Akyurt Facilities (\textbf{MGEO} Group)
\item Gölbaşı Facilities (\textbf{REHİS} Group)
\item ODTÜ TEKNOKENT (SATGEB Building) (Part of \textbf{HBT} Group)
\item ODTÜ TEKNOKENT (TİTANYUM Building) (Part of \textbf{HBT} Group)
\item İstanbul Facilities (Part of \textbf{SST} Group)
\end{itemize}



\vfill

\subsection{General Description of the Company}
\-
\indent ASELSAN is a company of Turkish Armed Forces Foundation, established in 1975 after Cyprus Peace Operation in order to meet the communication needs of the Turkish Armed Forces by national means. Currently 74,20\% of the shares are owned by the Foundation whereas the remaining 25,70\% runs in İstanbul Borsa stock market.

	As one of the largest defence industry companies of Turkey, ASELSAN's product portfolio includes communication and information technologies, radar and electronic warfare, electro-optics, avionics, unmanned systems, land, naval and weapon systems, air defence and missile systems, command and control systems, transportation, security, traffic, automation and medical systems\cite{aselsan} as can be seen from the  \textit{Figure~\ref{fig:As12}}. 
	
%	Today ASELSAN has become an indigenous products exporting company, investing in international markets through various cooperation models with local partners and listed as one of the top 100 defence companies of the world (Defense News Top 100).

\begin{figure}[H]
\center
\setlength{\unitlength}{\textwidth} 
%\includegraphics[width=0.9\unitlength]{organizasyon2}
\includegraphics[width=0.7\unitlength]{Aselsan1_2}
\caption{\label{fig:As12}Business Fields of ASELSAN }
\end{figure}

%	ASELSAN, together with the technology emphasis in its vision, has targeted to be a company that maintains its sustainable growth by creating value in the global market; preferred due to its competitiveness, trusted as a strategic partner, and caring for the environment and people.

	In 2018, ASELSAN is listed as 55\nth biggest defense company worldwide in DefenseNews' Top 100 list\cite{defense}. And as of July 2018, ASELSAN has 5364 employees. 63\% of them being engineer exact distribution of the employees can be seen at \textit{Figure~\ref{fig:calisan}}, while the academic distribution of the working engineers can be seen at \textit{Figure~\ref{fig:degree}}.

%	Together with the highly qualified engineering staff within more than 5000 employees, being the main driving factor of the company's success, ASELSAN allocates 6\% of its annual income for self-financed research and development activities.




\begin{figure}[H]
	\setlength{\unitlength}{\textwidth} 
	\centering
	\begin{subfigure}{.5\textwidth}
  		\centering
  		\includegraphics[width=0.48\unitlength]{calisan}
  		\caption{\label{fig:calisan}Distribution of the Employees }
	\end{subfigure}%
	\begin{subfigure}{.5\textwidth}
  		\centering
		\includegraphics[width=0.48\unitlength]{degree}
  		\caption{\label{fig:degree}Degree Levels of the Engineers }
	\end{subfigure}
\caption{\label{fig:calisandegree} Statistics about ASELSAN Employees   }
\end{figure}


\subsection{ASELSAN'S Vision \& Mission}

\subsubsection{Vision}
\- \indent
	To be a reliable, competitively preferred, environment-friendly and human conscious technology firm which preserves its sustainable growth in the global market via the values created for stakeholders, as well as serving its establishment purposes\cite{aselsan}.

\subsubsection{Mission}
\- \indent
	By focusing primarily on the needs of the Turkish Armed Forces; to provide high-value-added, innovative and reliable products and solutions to both local and foreign customers in the fields of electronic technologies and system integration; continuing activities in line with global targets as well as increasing brand awareness and contributing to the technological independence of Turkey\cite{aselsan}.

​
	
\vfill
	

\subsection{A Brief History of the Company}
\begin{itemize}
%\item \textbf{1976} 
%\subitem M. Hâcim KAMOY was assigned as the General Manager.
\item \textbf{ 1978 } : The first premises in Macunköy Facility were completed and the manufacturing operation started.
\item \textbf{ 1980 } : The first manpack and tank wireless radios were delivered to the Turkish Armed Forces.
%\item \textbf{ 1981 }
%\subitem The first hand-held radio and Bank Alarm Systems were designed. 
\item \textbf{ 1983 } : The first export was realized. 
\item \textbf{ 1982-1985 } : New products such as Field Telephones, Computer Controlled Central Systems and Laser Distance Measurement Appliances were included in the inventory. 
%\item \textbf{ 1986  }
%\subitem ASELSAN contributed to the power of Turkish Armed Forces with the Electronic Warfare and Data Terminal appliances it developed. 
\item \textbf{  1987 } : ASELSAN was included in a common project attended by 4 NATO countries for the manufacturing of Stinger Missile and started the required investment for the thick film hybrid circuit production. 
\item \textbf{  1988 } : ASELSAN produced the first avionic appliance for the F-16 program.
\item \textbf{ 1989 } : The first technology transfer to Pakistan was realized. Wireless radio production was started with ASELSAN license in NTRC facilities in Pakistan. 
\item \textbf{  1990 } : On date 21.05.1990, the ASELSAN shares were offered to the public and as of date 01.08.1990, the shares were started to be traded in IMKB (İstanbul Stock Exchange)
%\subitem ASELSAN was restructured in the 3 groups according to its fields of activity.
%\item \textbf{ 1991 }
%\subitem A Radar Technology Center was established in Aselsan with the SSIK 91-3 decision.
\item \textbf{ 1992 } : The Radar systems were included in the ASELSAN product range.
%\item \textbf{ 1992 }
%\subitem An Electro-Optical Technology Center was established in Aselsan with the SSIK 92-4 decision.
%\item \textbf{ 1994 }
%\subitem Studies with regard to design, assembly and commissioning works for Highway Emergency Assistance Communication Systems and Toll Collection Systems %and marketing of the same to foreign countries were started.
%\item \textbf{ 1995 }
%\subitem Project activities in main subjects such as Microelectronic, Guidance and Electro-Optical Group with the ongoing works and Hybrid Micro electronic, Inertial Navigational System, Infrared Guiding, Laser Guiding, Thermal Imaging Sensors, Passive Imaging Concentrators, Laser Generators and Sensors were realized.
%\item \textbf{ 1995 }
%\subitem Integration studies with regard to the applicability of electro-optical systems to different platforms and their more effective usage were realized and furthermore the production of ring laser gyroscope INS system was started.
\item \textbf{ 1996 } : The TASMUS agreement was executed.
\item \textbf{ 1997 } : ASELSAN 1919 Mobile Phone was launched to the market.
\item \textbf{ 1998 } : Thermal cameras, thermal weapon sight and thermal vision devices with target coordination addressing devices were submitted to the use of Turkish Armed Forces.
\item \textbf{ 1999 } : Agreements for Air Defence Early Warning and Command Control System, MILSIS Electronic Warfare and X-Band Satellite Communication System were executed.
%\item \textbf{ 2000 }
%\subitem Necip Kemal BERKMAN was assigned as the General Manager.
\item \textbf{ 2001 } : ASELSAN took over 72\% of the shares of ASELSAN MİKES A.Ş.
%\subitem The project for the serial production of KMS systems was executed. 
\item \textbf{ 2002 } : The equity capital of the company increased two and a half times compared to the previous year and reached the level of approximately one fourth of the aggregate resources.
%\subitem The Project for MWS-TU Missile Warning System and Leopard Volkan Fire Control System to be used in the Turkish Armed Forces Air Platforms was executed.
%\item \textbf{ 2003 }
%\subitem Agreements covering a long period for big projects such as SPEWS-II F-16 Electronic Warfare Auto Defense System, Military Police Integrated Communication and Information System were executed.
%\item \textbf{ 2004 }
%\subitem HEWS-CMDS CHAFF/FLARE shooter system Project was executed
\item \textbf{ 2005 } : HEWS, Helicopter Laser Warning Receiver system (LIAS) Project and Turkish Land Forces Avionic System Modernization Project was executed.
%\item \textbf{ 2006 }
%\subitem Cengiz ERGENEMAN was assigned to the General Manager position, Fuat AKÇAYÖZ was assigned as the Group President of Microwave and System Technologies, Dr. Faik EKEN was assigned as the Communication Devices Group President and KAHRAMANGİL was assigned as the Micro Electronic, Guidance and Electro-Optical Group President.
%\subitem ASELPOD Project was executed.
\item \textbf{ 2007 } : The construction of ASELSAN Integration Hall Building was completed and settlement activities were realized.
\item \textbf{ 2007 } : MILGEM war system supply project was executed.
\item \textbf{ 2008 } : ATAK agreement and Multi Band Digital Common Wireless Radio (ÇBSMT) Project were executed and ASELSAN delivered the first originally developed Air Defense Radar.
%\subitem In January 2008, Microwave and System Technologies Group Presidency was restructured as Defense System Technologies and Radar, Electronic Warfare and Communication Systems Group Presidency. Fuat AKÇAYÖZ was assigned to the position of Group President of Defense System Technologies and Ergun BORA was assigned to the position of Group President of Radar, Electronic Warfare and Communication Systems.
%\subitem In 2008, Coast Guard Command search and rescue Project, AKSAZ and FOCA Naval base under and surface surveillance and acquisition system (Yunus) Project, New Type Police Station Boat Project and JEMUS Kastamonu, Konya Wireless Radio system projects were executed. 
\item \textbf{ 2009 } : In 2009, four Research and Development Centrals were established, Leopard-1 Tank modernization was completed, MILGEM Warfare System 2nd Vessel Project, Ammunition Transfer system Project for Self-Propelled Howitzer (Fırtına- Storm) Ammunition vehicle and SAR / Reconnaissance System Supply Integration Project were executed.
%\subitem In 2009, STAMP and SOP system project for UAE, ADOP-2000 Fire Support System project, and the project for Land Located remote ED/ET capability gaining projects were executed.
%\item \textbf{ 2010 }
%\subitem In the year 2010, 112 Emergency Call Center was established in Antalya and Isparta, the Digital Trunk wireless radio system tender of İzmir Metropolitan Municipality was won and Tasmus-G 2nd Army Project deliveries were realized.
%\subitem In the year 2010, within the requirement by UAE, the subcontracting agreement was executed with Raytheon Company for the Patriot Missile System Antenna Mast Group products, ATMACA Electronic Systems development project, Pakistan Ministry of Defense Software Based Wireless Radio project, Naval Platform 3B Research Radar project, Self-propelled Air Defense Artillery and Fire Administration System Development project, 12 Air Defense Radar projects and 35 MM Towing Air Defense Artillery Modernization and Fragmentation Ammunition Development project were executed.
%\item \textbf{ 2011 }
%\subitem Following the manufacturing and plant acceptance tests of the Shipborne LPI Radar system ALPER (ASELSAN Low Power ECCM Radar) originally developed developed by ASELSAN, it was integrated to the TCG Heybeliada corvette within the scope of MILGEM Project, the Harbor Acceptance Tests were completed successfully and the first duty was started after the completion of the delivery.
%\subitem In the year 2011, MILGEM 1 Ship TCG HEYBELİADA Naval Acceptance Tests were completed successfully and was delivered to the navy. "AY Class Diesel-Electric Submarines Upgrade Project" was executed between . SSM, ASELSAN, STM and RAYTHEON companies. "Lower and Medium Altitude Air Defense Missile System Project Design and Development Period Agreement" was executed between SSM and ASELSAN. On date 12 April 2011, President Abdullah Gül visited the Macunköy Facility.
%\item \textbf{ 2012 }
%\subitem In May 2012 Necmettin BAYKUL was assigned as board of Directors. By the city hall, the name “Hacim KAMOY” founder of ASELSAN, has been given to the park nearby Macunköy facilities.
\item \textbf{  2012 } : Turkey’s first national Air Defense System “Pedestal Mounted Stinger System” which has been designed and produced by ASELSAN, and whose delivery took nearly 23 years, last 5 pieces has been delivered to Turkish Armed Forces.
\item \textbf{ 2013 } : ASELSAN has continued its climb for the aim of being one of the top 50 defense companies, and ranked 74th according to annual sales.
\item \textbf{ 2013 } : ASELSAN was the company who has participiated most at the 11th International Defence Industry Fair (IDEF 2013).	
%\subitem ASELSAN has won the “Leadership at Technology” award at the inovation week organized by Turkish Exporters’ Association. ASELSAN has also won “ Year 2013 Innovativeness Creativity Product Award”among the large companies with the SERHAT Counter Mortar Radar product at the event of TESİD Innovativeness Creativity Awards.

\end{itemize}

\vfill 

\subsection{The Organizational Chart of the Company}
\-
\indent
The organizational chart of ASELSAN can be seen in \textit{Figure~\ref{fig:orgc}}.

\begin{figure}[H]
\center
\setlength{\unitlength}{\textwidth} 
%\includegraphics[width=0.9\unitlength]{organizasyon2}
\includegraphics[width=0.9\unitlength]{organizasyon4}
\caption{\label{fig:orgc}The Organizational Chart of ASELSAN }
\end{figure}

	
	
\vfill

\section{Orientation and Mandatory Education}

\subsection{Orientation}
\- \indent
	My summer practice at ASELSAN started with an orientation program. The program lasted about one day mainly focused on the company and the work done there. Chapter 2 mostly summaries what was covered in the orientation. After the orientation, we were given necessary mandatory educations in order to be allowed to work inside the ASELSAN facilities.  

\subsection{Mandatory Educations}
\- \indent
	As required by the 4857/77  numbered law, all employers in the Republic of Turkey is obligated to train their employees in order to prevent the unnecessary work related accidents. In ASELSAN, we were given obligatory Occupational Safety and Health (OSH) Education and Electrostatic Discharge (ESD) Education to be able to work in the ASELSAN facilities safely.
	
	
	
\subsubsection{Electrostatic Discharge (ESD) Education }
\- \indent
	Electrostatic discharge (ESD) is the sudden flow of electricity between two electrically charged objects caused by contact, an electrical short, or dielectric breakdown. A buildup of static electricity can be caused by tribocharging or by electrostatic induction. The ESD occurs when differently-charged objects are brought close together or when the dielectric between them breaks down, often creating a visible spark \cite{esd}.

	A very casual example of electrostatic discharge can be given as lighting. However, not all ESD events are not as loudly or large-scale as lightnings. The less dramatic forms  may be neither seen nor heard, but they can still be large enough to cause damage to sensitive electronic devices. 

	ESD can cause harmful effects of importance in industry, including explosions in gas, fuel vapor and coal dust, as well as failure of solid state electronics components such as integrated circuits. In order to prevent this unwanted side effects of ESD, companies such as ASELSAN prefers to train their workers not just for their employee's health but also protect their product lines. 

\-\\

\subsubsection{Occupational Safety and Health (OSH) Education}
\- \indent
	ASELSAN as a company in Turkey is required to satisfy the conditions deternmined by the 6331 number Occupational Safety and Health (OSH) Education Law. 
	
	Occupational safety and health (OSH), also commonly referred to as occupational health and safety (OHS), is a multidisciplinary field concerned with the safety, health, and welfare of people at work. These terms also refer to the goals of this field, so their use in the sense of this article was originally an abbreviation of occupational safety and health program/department etc\cite{osh}.

	Occupational safety and health programs aims to foster a safe and healthy work environment. OSH may also protect co-workers, family members, employers, customers, and many others who might be affected by the workplace environment. 

	Just in 2014, 221.336 worker had an work accident imn Turkey and 494 of them suffered from work related diseases. 1.626 workers died due to this accidents according to ÇSGB\cite{6331}. The importance of Occupational Safety and Health (OSH) Educations comes from the fact that these deaths can be prevented if the necessary precautions are taken.

\section{Work Done at SP Company}
\- \indent
	I have performed my summer practice in the Test \& Process Design Department of the \textbf{HBT} Division. In this section, I will mainly explain what I have done in this department throughout my summer practice.
	 
	In my first days I was assigned to observe and participate the tests conducted at the Environmental   Test Laboratory. The test conducted there was mainly on ASELSAN 9661 Series Radios as well as other radio handsets and base stations. I mainly observed the these tests and participated in them as much as I could. I also observed and took part in the test about ASELSAN'S base station named ULAK. 

\-\\

\subsection{Electrical \& Mechanical Tests at the Environmental Tests Laboratory  }
\- \indent
	After my arrival to the Environmental Test Laboratory, I started observing the test mainly done on ASELSAN 9661 series V/UHF radio. Before getting into the details for the electrical tests, I will give some information about the device itself.   
	
\subsubsection{ASELSAN 9661 Radio Family}
\- \indent	
	The 9661 HF Radios are a software defined radio covering the HF 1.6-30MHz band. Software inside the radio supports various radio waveforms and EPM techniques. Beyond line of sight communication is made possible based on the HF technology via use of NATO STANAGs and Military Standards\cite{9661}. 


\begin{figure}[H]
	\center
	\setlength{\unitlength}{\textwidth} 
	\includegraphics[width=1.0\unitlength]{radio_type}
	\caption{\label{fig:radtyp}ASELSAN 9661 Radio Series\cite{9661} }
\end{figure}

	 While voice and data can be transmitted over a pre-set fixed frequency, it is also possible to employ an Automatic Channel Selection mechanism which determines the usable frequency for communication.

	

%\begin{figure}[H]
%	\center
%	\setlength{\unitlength}{\textwidth} 
%	\includegraphics[width=1.0\unitlength]{radios}
%	\caption{\label{fig:rads}ASELSAN 9661 Radio Series }
%\end{figure}

	
	9661 HF Radio family has three configurations for Manpack, Vehicle and Fixed Station usage. 20W can be used for Manpack and Vehicle configurations and 150 W can be used for Vehicle and Fixed Station configurations. The product line can be seen at \textit{Figure~\ref{fig:radtyp}}. As I spent my time at the laboratory, the test I observed were for the 100 W Vehicular/Base Station models. 

\- \\[2cm]




\subsubsection{Test Devices}
\- \indent
	Just before going into details about some of the tests conducted at the laboratory, I will give brief explanation about the devices used as tests are conducted. General test devices used can be seen at \textit{Figure~\ref{fig:components}}.

\begin{figure}[H]
	\center
	\setlength{\unitlength}{\textwidth} 
	\includegraphics[width=0.75\unitlength]{components}
	\caption{\label{fig:components}General Test Devices \cite{9661,specan,keysight,mult}}
\end{figure}

\paragraph{Spectrum Analyzer} 
\- \indent
	Spectrum analyzer was one of the most unfamiliar devices for me during the test processes. Thus, I would like to give some information about the device itself. 
	
	A spectrum analyzer measures the magnitude of an input signal versus frequency within the full frequency range of the instrument. The primary use is to measure the power of the spectrum of known and unknown signals. The input signal that a spectrum analyzer measures is electrical; however, spectral compositions of other signals, such as acoustic pressure waves and optical light waves, can be considered through the use of an appropriate transducer. 

	By analyzing the spectra of electrical signals, dominant frequency, power, distortion, harmonics, bandwidth, and other spectral components of a signal can be observed that are not easily detectable in time domain waveforms. These parameters are useful in the characterization of electronic devices, such as wireless transmitters.

	The display of a spectrum analyzer has frequency on the horizontal axis and the amplitude displayed on the vertical axis. To the casual observer, a spectrum analyzer looks like an oscilloscope and, in fact, some lab instruments can function either as an oscilloscope or a spectrum analyzer. An example for the analyzer can also be seen at \textit{Figure~\ref{fig:components}}.

\paragraph{Attenuators} 
\- \indent
	RF attenuators are a universal building block within the RF design arena. RF attenuators can be fixed, switched or even continuously variable.

Dependent upon their type, they can be designed using just resistors, they may need a switch, either mechanical or solid state, or they may use diodes to make them continuously variable over a given range.

	As the name implies RF attenuators reduce the level of the signal, i.e. they attenuate the signal.

\begin{figure}[H]
	\center
	\setlength{\unitlength}{\textwidth} 
	\includegraphics[width=0.5\unitlength]{attn}
	\caption{\label{fig:attn}Some Attenuator Examples }
\end{figure}

	This attenuation may be required to protect a circuit stage from receiving a signal level that is too high. Also an attenuator may be used to provide an accurate impedance match as most fixed attenuators offer a well-defined impedance, or attenuators may be used in a variety of areas where signal levels need to be controlled.

	There are many used for these RF attenuators and although these may not seem obvious initially when asking what is an attenuator, they are widely used in RF applications.




\subsubsection{Electrical Configuration/Performance Tests}
\- \indent
	As I spent my time at laboratory, I witnessed many steps of finalized product test, environmental tests and so on. In this section, I will mention some of these electrical tests in more detail. These tests were actually standardized by the TIA (Telecommunications Industry Association) in TIA/EIA 102 or more commonly known as Project 25.

\- \\

\paragraph{Reference Sensitivity Test}
\- \indent
	The Reference sensitivity is the level of receiver input signal at a specified frequency with specified modulation that will result in the standard BER (Bit Error Rate) at the receiver detector according to TIA Standart called Project 25 \cite{P25}.

\-\\

\begin{figure}[H]
	\center
	\setlength{\unitlength}{\textwidth} 
	\includegraphics[width=1.0\unitlength]{refsens}
	\caption{\label{fig:refsens}Reference Sensitivity Test System Diagram }
\end{figure}

\- \\

	To measure the reference sensitivity, test set-up at the \textit{Figure~\ref{fig:refsens}} must be established. After adjusting the input level to achieve   standard bit error rate at the output when measured at least a 250 ms time interval, the reference sensitivity will be signal level at the RF Signal Generator.

\- \\[1.5cm]

\paragraph{Residual Audio Noise Ratio Test}
\- \indent
	The audio output distortion is the voltage ratio, usually expressed as a percentage of the rms value of the undesired signal to the rms value of the complete signal, at the output of the receiver according to Project 25 Standart\cite{P25}.

\begin{figure}[H]
	\center
	\setlength{\unitlength}{\textwidth} 
	\includegraphics[width=0.75\unitlength]{resaudtest}
	\caption{\label{fig:resaudtest}Residual Audio Noise Ratio Test System Diagram }
\end{figure}

	To measure the residual audio noise ratio, test set-up at the \textit{Figure~\ref{fig:resaudtest}} must be established. Firstly, while applying a standard input signal at the standard input signal level to the receiver input terminals, the receiver for rated audio frequency output power should be adjusted. Then, the audio output level is recorded as $V_{Ref}$.
	
	After doing so, the input signal should be changed to the standard silence test pattern, in other words no signal should be supplied to the test system. The audio output level at this scenario is recorded as $V_{S}$.
	
	Lastly, the signal generator is removed to mute the receiver and measure the receiver audio output. This audio level is recorded as $V_{Mute}$. The residual audio noise ratio can be calculated as follows:
	
$$	Residual~Audio~Noise~Ratio~(Silence)~=~20~log_{10}(\frac{V_{Ref}}{V_{S}}) 	$$

$$	Residual~Audio~Noise~Power~(Mute)~=~10~log_{10}(1000\frac{V^2_{Mute}}{R_{Load}}) 	$$


%\- \\[1cm]

\paragraph{RF Output Power Test}
\- \indent
	The RF output power of a transmitter for this service is the power available at the output terminals of the transmitter when the output terminals are connected to the standard transmitter load according to Project 25 Standart\cite{P25}.

	To measure the RF output power, test set-up at the \textit{Figure~\ref{fig:rfouttest}} must be established
	
	
\begin{figure}[H]
	\center
	\setlength{\unitlength}{\textwidth} 
	\includegraphics[width=1.0\unitlength]{rfouttest}
	\caption{\label{fig:rfouttest}RF Output Power Test System Diagram }
\end{figure}

%\paragraph{Transmitter FM Hum/Noise Ratio Test}
%\- \indent


%\begin{figure}[H]
%	\center
%	\setlength{\unitlength}{\textwidth} 
%	\includegraphics[width=1.0\unitlength]{transnoise}
%	\caption{\label{fig:transnoise}Transmitter FM Hum/Noise Ratio Test System Diagram }
%\end{figure}





\paragraph{Electrical Audio Performance Test}
\- \indent
	The electrical audio performance is the degree of closeness to which the audio path of the transmitter follows a prescribed characteristic according to Project 25 Standart\cite{P25}.

	To measure the RF output power, test set-up at the \textit{Figure~\ref{fig:elecaudio}} must be established
	
\begin{figure}[H]
	\center
	\setlength{\unitlength}{\textwidth} 
	\includegraphics[width=1.0\unitlength]{elecaudio}
	\caption{\label{fig:elecaudio}Electrical Audio Performance Test System Diagram }
\end{figure}

\paragraph{Modulation Fidelity Test}
\- \indent
	Modulation fidelity is defined as the degree of closeness to which the modulation follows the desired ideal theoretical modulation by the P25(TIA/EIA-102) standard\cite{fidelity}. This is a very important measurement for P25 C4FM modulation as it is an indication of the quality of the signal being transmitted by the radio. Before going into the details of the
modulation fidelity measurement, I will briefly explain the type
of modulation that this measurement analyses.

\subparagraph{C4FM Modulation} \-\\
\- \indent
	P25 uses a type of modulation called C4FM, which is an acronym for “compatible 4 level frequency modulation”\cite{fidelity}. Basically, it is a special type of 4FSK modulation developed for the TIA/EIA-102 standard. P25 uses this type of modulation to transmit digital information in the form of digital “1’s” and “0’s”. 4FSK uses four different frequency “states” or “deviation points” to indicate a “symbol”. This symbol then equates to 2 bits of data as one of the four frequency shifts. The frequency shifts that correspond to each 2 bits of data are shown in \textit{Table~\ref{tab:modfide}}.
	
\begin{table}[H]
  \centering
 
    \begin{tabular}{c|c|c}
       $$Bits$$ & $$Symbols$$ & $$C4FM Deviation$$ \\ \hline
       00 & +3 & +0.6 kHz  \\ \hline
       01 & +1 & +1.8 kHz  \\ \hline
       10 & -1 & -0.6 kHz  \\ \hline
       11 & -3 & -1.8 kHz  
      
  \end{tabular}
  \caption{P25 C4FM Frequency Deviation States}
  \label{tab:modfide}
\end{table}

	This information is sent at the “symbol rate”. For P25, this symbol rate is transmitted 4800 times per second. This results  with a bit rate of 9600 bits per second. 
		
	As part of modulation fidelity, it is desired to measure the deviation of each of the symbols that the radio under test generates and compare them with the ideal four deviation points indicated in \textit{Table~\ref{tab:modfide}}. This measurement will actually result in producing three important values that together will be indicators of the modulation fidelity of the radio under test.

\- \\ \- \\

\subparagraph{Frequency Error} \- \\
\- \indent
	The first value that we can calculate from the measured deviation of each of the symbols is the frequency error. Frequency error in this measurement refers to RF carrier frequency error.
		
	To understand, the relationship between the four frequency deviations used in C4FM and RF carrier frequency error can be thought. A carrier frequency error would tend to shift all four of the deviations by the same amount. A positive frequency error would move all four of the deviation points in the positive direction. For convince, a real world scenario is used to illustrate the situation. In \textit{Table~\ref{tab:modfide2}} below, a 100 Hz frequency error might give the following results:


\begin{table}[H]
  \centering
 
    \begin{tabular}{c|c|c}
       $$Bits$$ & $$Symbols$$ & $$C4FM Deviation$$ \\ \hline
       00 & +3 & +0.7 kHz  \\ \hline
       01 & +1 & +1.9 kHz  \\ \hline
       10 & -1 & -0.64 kHz  \\ \hline
       11 & -3 & -1.9 kHz  
      
  \end{tabular}
  \caption{Example of a Measured P25 Frequency Deviation
States}
  \label{tab:modfide2}
\end{table}

\- \\




	In ideal world, the RF carrier error would shift evey symbol by the same frequency amount but in real world, there are other effects that may affect symbol deviation.
	
	To find the frequency error, the average frequency deviation for each of the four symbols should be found, and then the average of these four deviation points should be calculated. For the given example, from the \textit{Table~\ref{tab:modfide2}}, the average would be:
	
	$$	\frac{0.7+1.9-0.64-1.9}{4}~=~0.015~kHz~=~15~Hz	$$


\subparagraph{Deviation} \- \\
\- \indent
	To calculate the deviation, the four average values calculated for each symbol that were also used to find frequency errors are used. By the help of these numbers, \textit{Table~\ref{tab:deviat}} can be constructed.

\begin{table}[H]
  \centering
 
    \begin{tabular}{c|c|c|c}
       $$Info Bits$$ & $$Ideal Deviations$$ & $$\specialcell{ Average deviation \\ after subtracting out the  \\ frequency error }$$  & $$ Deviation ratio $$ \\ \hline
       00 &  +0.6 kHz & +0.72 kHz & $\frac{0.7}{0.6}~=~1.16$ \\ \hline
       01 &  +1.8 kHz & +1.9 kHz  & $\frac{1.9}{1.8}~=~1.05$ \\ \hline
       10 &  -0.6 kHz & -0.64 kHz & $\frac{0.64}{0.6}~=~1.06$ \\ \hline
       11 &  -1.8 kHz & -1.9 kHz & $\frac{1.9}{1.8}~=~1.06$ 
      
  \end{tabular}
  \caption{P25 C4FM Frequency Deviation Ratio}
  \label{tab:deviat}
\end{table}

	From the \textit{Table~\ref{tab:deviat}}, the deviation for this example can be calculated a follow:
	
	$$	Avergae~Deviation~Ratio~= ~\frac{1.16+1.05+1.06+1.06}{4}~=~1.085	$$
	


%	The fidelity is determined by from observations of the signal at the output of an integrate and dump filter, that is preceded by an FM demodulator. The filtered FM modulation trajectories for C4FM and CQPSK are different at all points in time except at the symbol decision points. The modulation fidelity is measured by determining the rms difference between the actual signal and the ideal C4FM deviation for the transmitted symbols as information.
	
	
	
	For general calculations, let $s_K$ represents the C4FM deviation of the transmitted symbols, and $z_K$ represents the detected signals at $t_K$ sampling instants.
	
	The transmitter can be modelled as
	
	$$ z_K~=~C_O+C_L*(s_K~+~e_K) $$
	
	where $C_O$ is a constant representing carrier frequency offset, the $C_L$ is another constant called deviation errors that is resulting from gain errors in the transmitter's modulator baseband signal processing. And $e_K$ is called residual deviation error.
	
	The sum-square deviation error is then:
	
$$	\sum_{k=MIN}^{k=MAX} {|e_K|}^2 = \sum_{k=MIN}^{k=MAX} {|\frac{z_K - C_O}{C_L}-s_K|}^2	$$
	
	$C_0$ \& $C_L$ should be chosen to minimize the sum-suquare deviation.
	
	
\begin{figure}[H]
	\center
	\setlength{\unitlength}{\textwidth} 
	\includegraphics[width=1.0\unitlength]{modfide}
	\caption{\label{fig:modfide}Modulation Fidelity Test System Diagram }
\end{figure}


	For further test, test set-up at \textit{Figure~\ref{fig:modfide}} can be used. 





\subsubsection{Tests on ULAK 4.5G Base Station  }
\- \indent
	Besides 9661 Series Radios, multiple radio hand-sets and the 4.5G Base Station were tested at Environmental Test Laboratory. Similar electrical tests applied to the 9661 radio series were partially applied to the these base stations. Since I was not able spent enough time on the tests of the base station, I will quickly introduce the base station itself and pass the next section. 

\paragraph{4.5G Macrocell Base Station}
\- \indent
	Based on Release 10 and Release 11 standards published by 3GPP, ULAK Macrocell Base Station is designed to support both Release 12 and Release 13 standards and is designed to work on different frequency bands for use in Commercial or Public Safety networks\cite{ulak} by ASELSAN A.Ş. and ULAK Haberleşme A.Ş which is owned by SSTEK Defense Industry Technologies and ASELSAN.  The base station can be seen at \textit{Figure~\ref{fig:ulak}}.

\begin{figure}[H]
	\center
	\setlength{\unitlength}{\textwidth} 
	\includegraphics[width=1.0\unitlength]{ulak}
	\caption{\label{fig:ulak}ULAK 4.5G Base Station\cite{ulak} }
\end{figure}	
	
	
\-\\

\subsection{Outer Space Simulations \& Tests using TVAC  }
\- \indent
	As I worked in the Test \& Process Design Department, I did not spend my whole time at Environmental Test Laboratory, I also spent some of my time at the Space Simulation Laboratory of the Department. The laboratory was responsible for the outer space test of the electrical components of he TURKSAT 6-A project as I made summer practice. Before going into detail, I will give brief information about the thermal chambers used there.
	
\subsubsection{Thermal Vacuum Chamber (TVAC) }
\- \indent
	A vacuum chamber is a rigid enclosure from which air and other gases are removed by a vacuum pump. This results in a low-pressure environment within the chamber, commonly referred to as a vacuum. A vacuum environment allows researchers to conduct physical experiments or to test mechanical devices which must operate in outer space (for example) or for processes such as vacuum drying or vacuum coating. Chambers are typically made of metals which may or may not shield applied external magnetic fields depending on wall thickness, frequency, resistivity, and permeability of the material used. Only some materials are suitable for vacuum use\cite{tvac}.


\begin{figure}[h!]
	\center
	\setlength{\unitlength}{\textwidth} 
	\includegraphics[width=1.0\unitlength]{tvac}
	\caption{\label{fig:tvac}An Example Thermal Vacuum Chamber \cite{tvac}}
\end{figure}

	A type of these vacuum chambers widely used in the spacecraft engineering field is a thermal vacuum chamber, which can simulate the thermal environment at which a spacecraft would be operating in space. In other words,	a thermal vacuum chamber is a vacuum chamber in which the radiative thermal environment is controlled. One example for this type of chamber can be seen at \textit{Figure~\ref{fig:tvac}} which is used in NASA.


	There are two thermal vacuum chambers with different capacities in the ASELSAN HBT Facilities. They are used mainly for the outer space simulation tests of the electronics parts of the TURKSAT 6A Project. The chambers can also be used by other project partners such as TAI and TURKSAT if needed.
	
\subsubsection{Tests \& Simulations Using Thermal Vacuum Chamber (TVAC) }
\- \indent
	To simulate the outer space, the pressure in the chamber must be reduced to $10^{-5}~mBar$ or below to satisfy the requirements for the test. This was done by pulling the oxygen and other major atmospheric gases and pumping liquid nitrogen to the system to push leftover gases. Liquid Nitrogen is also used to reach the required minimum temperature for the test.
	
	According to the device under test's physical properties, two ways are used to reach the desired maximum temperature. One of them is using spacial thermal plates below the device. If the device should not be in contact with directly heat source second method for heating can be used.  That is to use the infrared heaters inside the chambers.
	
\begin{figure}[H]
	\center
	\setlength{\unitlength}{\textwidth} 
	\includegraphics[width=1.0\unitlength]{tvac-cycle}
	\caption{\label{fig:tvac-cycle}An Example Temperature Cycle }
\end{figure}


	 To initialize the test, the atmospheric pressure inside the vacuum is reduced to $10^{-5}~mBar$ or below at room temperature. The device is expected to operate in such pressure levels, therefore, the pressure level is conserved until the last cycle by the PID controller inside the chamber. After the necessary pressure requirements is met, the temperature is set according to the customer's will, in our case customer was TURKSAT A.Ş. as owner of TURKSAT 6A. An example test cycle for testing the device under test can be seen at \textit{Figure~\ref{fig:tvac-cycle}}. Between each temperature change, the temperature inside the chamber is left unchanged up to six hours to ensure the stability of the operation of the device at that temperate. This can be done by taking the necessary measurement with half hours intervals. If the data are consistent with previous, temperature will be changed for next cycle. Unfortunately, I was not allowed the take screenshots for the test results or took pictures of other test steps,  I will not be able to give further information about this topic.
		

\subsection{Research on Components of ULAK 4.5G Base Station  }
\- \indent
	Mainly on the second half of my internship, I also spent my time at the department itself besides the laboratories. In this part of my summer practice, I was responsible for making research on the components of the ULAK Base Station for possible upgrade for the device. Due to regulations, I will give some examples about the work done at this part of summer practice. 


\subsubsection{Misidentified Components}
\- \indent
	As I spent my time at department itself, I was responsible for two types of component research. One type of these researches was to identify the problem on the newly arrived printed circuit boards for the base station. Since I was not permitted to take the photographs of these components, I will give some details without actual photos for the components. 	
	
	Two give a specific example, I will give detail about one of this problems. In one part of the latest arrived printed circuit boards that is produced by the subcontractor, a very minor variation was not noticed until it came to ASELSAN. When I examine the PCB closely, I noticed the difference in the small IC. On the top of the IC, it was labeled as \textbf{C1H} instead of \textbf{C1R}. Top views of both of these components can be seen at \textit{Figures~\ref{fig:2746top}~and~\ref{fig:2712top}} respectively.  
	 
	

	
\begin{figure}[H]
	\setlength{\unitlength}{\textwidth} 
	\centering
	\begin{subfigure}{.5\textwidth}
  		\centering
  		\includegraphics[width=0.24\unitlength]{2746_ust}
  		\caption{\label{fig:2746top}Top View of the 2746TB}
	\end{subfigure}%
	\begin{subfigure}{.5\textwidth}
  		\centering
		\includegraphics[width=0.24\unitlength]{2712_ust}
  		\caption{\label{fig:2712top}Top View of the 2712TB}
	\end{subfigure}
\caption{\label{fig:274612top} Top View of the UPC2746TB \& UPC2712TB ICs  \cite{2746,2712} }
\end{figure}	

	In the following subsection, I will mention critical differences between these ICs in detail and explain why the circuit was not functioned properly due to this mistake.

\-\\[2.5cm]

\paragraph{NEC UPC2746TB vs NEC UPC 2712TB}
\- \indent
	Although both ICs are Silcom MMIC Wideband Amplifiers manufactured using same technologies, UPC2712TB is designed for use as buffer amplifiers in DBS tuners\cite{2712} whereas the UPC2746TB is designed for use as buffer amplifiers in mobile communication applications such as Celular ,PCS, Cordless handsets and WLAN transceivers\cite{2746}. However, since both ICs are housed in a 6-pin super minimold (SOT-363) package and the pins are functioning exactly the same, it is rather easy making mistakes using one instead of another.
	
	
%\begin{table}[H]
%  \centering
% 
%    \begin{tabular}{c|c|c}
%       \backslashbox{$A$}{$a$} & $$B$$ & $$C$$ \\ \hline
%       \multirow{2}{*}{1} & 2 & 3  \\ \cline{2-3}
%        & 3 & 4  \\ \hline
%       3 & \multicolumn{2}{c}{4}  \\ \hline
%       4 & 5 & 6  
%      
%  \end{tabular}
%  \caption{table}
%  \label{tab:table}
%\end{table}

\begin{table}[H]
  \centering
 
    \begin{tabular}{c|c|c}
       $$Pin~No$$ & $$Pin~Name$$ & $$Description$$ \\ \hline
       1 & Input  & Signal input pin  \\ \hline
       4 & Output & Signal output pin  \\ \hline
       6 & $V_{CC}$ & Power supply pin  \\ \hline
       2\&3\&5 & GND & Ground pin
 	\end{tabular}
  \caption{Pin Connections for UPC2746TB \& UPC2712TB}
  \label{tab:pins}
\end{table}

	To examine differences between 2746 \& 2712 , internal equivalent circuits for the respected ICs can be seen from the \textit{Figures~\ref{fig:2746cct}~and~\ref{fig:2712cct}}. As mentioned earlier, in both devices, pins are used for similar functions and these functions can be seen at \textit{Figures~\ref{tab:pins}}.  
	

	
\begin{figure}[H]
	\setlength{\unitlength}{\textwidth} 
	\centering
	\begin{subfigure}{.5\textwidth}
  		\centering
  		\includegraphics[width=0.3\unitlength]{2746_cct}
  		\caption{\label{fig:2746cct}Equivalent Circuit of the 2746TB}
	\end{subfigure}%
	\begin{subfigure}{.5\textwidth}
  		\centering
		\includegraphics[width=0.3\unitlength]{2712_cct}
  		\caption{\label{fig:2712cct}Equivalent Circuit of the 2712TB}
	\end{subfigure}
\caption{\label{fig:274612cct} Absolute Maximum Ratings for the UPC2746TB \& UPC2712TB ICs   }
\end{figure}	


	Although, in many aspects these amplifiers seems like an alternative to each other, main differences occurs in supply parameters and performance characteristics of these ICs.
	
\begin{table}[H]
  \centering
 
    \begin{tabular}{c|c|c|c|c}
       $$Symbols$$ & $$Parameters$$ & $$Units$$  & $$UPC2746$$  & $$UPC2712$$ \\ \hline
       $V_{CC}$ & Supply Voltage & V & 6 &  4 \\ \hline
       $P_{IN}$ & Input Power & dBm &  +10 &  0 \\ \hline
       $P_T$ & Total Power Dissipation & mW & 200 & 200 \\ \hline
       $T_{OP}$ & Operating Temperature & $^{\circ}C$ & -45 to +85  & -45 to +85 \\ \hline
       $T_{STG}$ & Storage Temperature &  $^{\circ}C$ & -55 to +150 & -55 to +150 
 	\end{tabular}
  \caption{Absolute Maximum Ratings for UPC2746TB \& UPC2712TB}
  \label{tab:absmax}
\end{table}
	
	When the \textit{Tables~\ref{tab:absmax} and \ref{tab:reccomb} } are examined, it can be clearly seen that voltage supplie values of these components differ each other dramatically. For instance, as the UPC2712 operates perfectly at 5V, the UPC2746 can not even operate since its absolute maximum voltage supply value is 4V as can be seen from the \textit{Table~\ref{tab:absmax}}.  	
	
	
\begin{table}[H]
  \centering	
	\begin{tabular}{c|c|c|c|c|c|c}
       $Symbol$ & $$Parameter$$ & $$Units$$ & $$IC$$ & $$MIN$$ & $$TYP$$ & $$MAX$$ \\ \hline
       \multirow{2}{*}{$V_{CC}$} & \multirow{2}{*}{Supply Voltage} & \multirow{2}{*}{V} & UPC2746 & 2.7 & 3.0 & 3.3  
       \\ \cline{4-7}  & & & UPC2712 & 4.5 & 5.0 &  5.5  \\ \hline
       \multirow{2}{*}{$T_{OP}$} & \multirow{2}{*}{Operating Temperature} & \multirow{2}{*}{$^{\circ}C$} & UPC2746 & -40 & +25 & +85  
       \\ \cline{4-7}  & & & UPC2712 & -40 & +25 & +85 
       \end{tabular}
       \caption{Recommended Operating Conditions for UPC2746TB \& UPC2712TB}
  \label{tab:reccomb}
\end{table}

	Another important difference between these components is about their performance curves ,or in other words, gain vs frequency plots. As can be seen from the \textit{Figures~\ref{fig:2746perf} and \ref{fig:2712perf}} respectively, the gains at high frequencies such as 3 GHz differs significantly even though the gain at lower frequencies such as 0.3 GHz are comparable. 
	
	For that reasons, it was not possible to use the newly arrived printed circuit boards, and they were returned to the subcontractor.


\begin{figure}[H]
	\setlength{\unitlength}{\textwidth} 
	\centering
	\begin{subfigure}{.5\textwidth}
  		\centering
  		\includegraphics[width=0.45\unitlength]{2746_perf}
  		\caption{\label{fig:2746perf}Performance Curve for the 2746TB }
	\end{subfigure}%
	\begin{subfigure}{.5\textwidth}
  		\centering
		\includegraphics[width=0.45\unitlength]{2712_perf}
  		\caption{\label{fig:2712perf}Top View of the UPC2712TB IC}
	\end{subfigure}
\caption{\label{fig:274612perf} Typical Performance Curves for the UPC2746TB \& UPC2712TB ICs   }
\end{figure}	


\subsubsection{Alternate Components}
\- \indent
	As I mentioned earlier, I was responsible for two types of component research at the department itself. One type of these researches was to identify the problem on the newly arrived printed circuit and the second type of these researches was to search and find possible components to replace the existing one. Although these components mostly worked perfectly, they needed to be replaced due to other factors. 	
	
	In the following subsection, I will give brief detail about one of these possible changes.
	
\paragraph{Change of Backplane Connector}
\- \indent
	Due to some supply problems of the backplane connector used to connect the motherboard and the daughter-board of base station, the production of the base station was not at the desired level. For that reasons, I was asked to find a replacement for the connector used.
	
	The connector used was from ERmer ZDplus backplane connector family of ERNI. Main advantage of it was that it was capable of transferring data over 20 Gbit/s which more than most of the connector in the market.
	
\begin{figure}[H]
	\center
	\setlength{\unitlength}{\textwidth} 
	\includegraphics[width=0.9\unitlength]{zpack}
	\caption{\label{fig:zpack}TE Connectivity Z-Pack HM-Zd Plus Backplane Connectors \cite{teconnec}}
\end{figure}


%\begin{figure}[H]
%	\center
%	\setlength{\unitlength}{\textwidth} 
%	\includegraphics[width=1.0\unitlength]{zpack2}
%	\caption{\label{fig:tvac-cycle}An Example Temperature Cycle }
%\end{figure}

	 

	Due to its similarities in physical properties, I proposed that The Z-Pack HM-Zd Plus backplane connector family from TE Connectivity is good alternative for the used one.
	
	Both of the connectors can be seen at \textit{Figures~\ref{fig:zpack} and \ref{fig:ermetzd}}.\cite{ermetzd}  \cite{teconnec} 
%\begin{figure}[H]
%	\center
%	\setlength{\unitlength}{\textwidth} 
%	\includegraphics[width=1.0\unitlength]{ermetzd}
%	\caption{\label{fig:tvac-cycle}An Example Temperature Cycle }
%\end{figure}

\begin{figure}[H]
	\center
	\setlength{\unitlength}{\textwidth} 
	\includegraphics[width=0.9\unitlength]{ermetzd2}
	\caption{\label{fig:ermetzd} ERNI Ermet ZDplus Backplane Connectors \cite{ermetzd} }
\end{figure}



\- \vfill
%\begin{enumerate}
%\item --
%\item --
%\item --
%\item --
%\item --
%\item --
%\end{enumerate}

%\begin{figure}[H]
%\centering
%%<\includegraphics[scale=1.0]{todo2.jpg}
%\caption{\label{fig:------} ------ ------  }
%\end{figure}
								
	
%\begin{table}[H]
%  \centering
 
%    \begin{tabular}{c|c|c}
%       &$$------$$ & $$------------$$ \\ \hline
%       1 & -- & ---  \\ \hline
%       2 & -- & ---  \\ \hline
%       3 & -- & ---  \\ \hline
%       4 & -- & ---  \\ \hline
%       5 & -- & ---  \\ \hline
%       6 & -- & --- 
%      
%  \end{tabular}
%  \caption{------}
%  \label{tab:------}
%\end{table}
	
	
%\begin{lstlisting}[style=CStyle]


%----


%\end{lstlisting}

%\vfill

%\begin{lstlisting}[language=Matlab]

%----

%\end{lstlisting}



\section{Conclusion}
\-\indent 
	I completed my summer practice in ASELSAN A.Ş.(ASELSAN Electronics Industry and Ticaret A.Ş.) in supervision of Pınar Kırıkkanat, an electronics engineer in ASELSAN, in Yenimahalle/Ankara. It was quite experiential time for me. Throughout my summer practice, I learned many things about professional work life. 

	Firstly, I understood the importance of mandatory educations like occupational safety and health education thanks to given educations by ASELSAN. After the educations, I was  sent to my division, where I would performed my summer practice.
	
	In the first half of my internship, I was given time to observe, learn and participate the mechanical and electrical test conducted at our division. Mainly on ASELSAN 9661 Series Radios, I mostly observed and participated on the environmental tests of the equipments produced at the Communication \& Information Technologies Vice Presidency, known as HBT. .
	
	In the second part of my internship, I have made some research for the ULAK Base station of the ASELSAN regarding its components. Since the work done at this stage can be mostly considered as classified information, I mentioned the basics of what I have done in this part.
	
	In this, report, I started with an introduction, that covers what was done in my summer practice. Then, I continued with a company description section in which the general description about ASELSAN is given. After that part, the work done in my summer practice is explained. Lastly, I finished the report with an conclusion part. 
			
	Finally, I recommend my summer practice company for other students who want to experience the professional work life at ASELSAN. 

\-\vfill 


\section{References}

\begingroup
\renewcommand{\section}[2]{}%
%\renewcommand{\chapter}[2]{}% for other classes
\begin{thebibliography}{}

%\bibitem{gitsun} Temurtas Halil,
%	\textit{Sun Tracker System},
%	Bitbucket repository,(2017),
%	https://bitbucket.org/temurtas/pi/
	
\bibitem{defense} DefenseNews Top 100 Defense Companies List 2018,(n.d.), 
Retrieved September 1, 2018, from https://people.defensenews.com/top-100/
	
%\bibitem{attinf} RF Attenuator Basics \& Tutorial. ,(n.d.), 
%Retrieved September 2, 2018, from https://www.radio-electronics.com/info/rf-%technology-design/attenuators/rf-attenuators-basics-tutorial.php

%\bibitem{attpic}	Attenuators, (2016, December 03),
%Retrieved August 28, 2018, from https://nelsoncreated.com/products/attenuators-2

%\bibitem{powref}	C,
%	http://www.livingston-products.com/products/pdf/105517\_1\_en.pdf

\bibitem{audany} Rohde \& Schwarz GmbH \& Co KG., (n.d.), Brochures and Data Sheets for Product UPV, Retrieved September 15, 2018, from https://www.rohde-schwarz.com/us/brochure-datasheet/upv/

\bibitem{tvac} Jenner, L. (2017, May 31), NASA's Apollo-Era Test Chamber Now James Webb Space Telescope Ready, 
Retrieved August 24, 2018, from https://www.nasa.gov/feature/goddard/2017/nasas-apollo-era-test-chamber-now-james-webb-space-telescope-ready

\bibitem{6331} 6331 Sayılı İş Sağlığı Kanunu, (2016, April),
Retrieved August 14, 2018, from https://www.csgb.gov.tr/media/4574/kitap01.pdf

\bibitem{fidelity} Project 25, (2018, September 20),
Retrieved October 2, 2018, from https://en.wikipedia.org/wiki/Project\_25

\bibitem{aselsan} ASELSAN About Us, (n.d.),
Retrieved July 27, 2018, from https://www.aselsan.com.tr/en-us/about-us/Pages/Default.aspx

\bibitem{9661} ASELSAN 9661 Series Radio Family, (n.d.), 
Retrieved August 4, 2018, from https://www.aselsan.com.tr/en-us/capabilities/cpb1/Brochures/9661\_HF/9661 HF Gound Radio Family.pdf

\bibitem{ulak} ULAK Base Station, (n.d.), 
Retrieved August 4, 2018, from https://www.aselsan.com.tr/en-us/capabilities/public-safety-communication-systems/mobile-communication-technologies-ulak-4-5g-5g/ulak-commercial-and-public-safety-4-5g-macrocell-base-station


\bibitem{specan} Rohde-Schwarz Spectrum Analyser, (n.d.), 
Retrieved August 4, 2018, from https://cdn.rohde-schwarz.com/pws/product/f\_1/fsu/FSU\_front\_stage\_landscape.jpg

\bibitem{keysight} Keysight Radio Test Reference Solution, (n.d.), 
Retrieved August 19, 2018, from http://rfmw.em.keysight.com/mod/pdf/solutions\_radio\_test\_ref\_soln\\\_presentation.pdf

\bibitem{mult} Keysight PN-3458A Digital Multimeter, (n.d.), 
Retrieved August 19, 2018, from Retrieved August 19, 2018, from https://www.keysight.com/en/pdx-2905513-pn-3458A/digital-multimeter-8-digit?cc=TR\&lc=eng


\bibitem{P25} Land Mobile Radio Transceiver Performance Recommendations , (2010,January), Retrieved August 4, 2018, from https://www.qsl.net/kb9mwr/projects/dv/apco25/TIA-102.CAAB-C-2010.pdf



\bibitem{ermetzd} ERNI Ermet ZDplus Backplane Connectors , (n.d.) , Retrieved August 27, 2018, from https://www.erni.com/fileadmin/user\_upload/Downloads/Products/\\ERmet\_ZD/ERNI-ERmetZDplus-EN.pdf

\bibitem{teconnec} TE Connectivity Z-Pack HM-Zd Plus Backplane Connectors , (n.d.), Retrieved August 4, 2018, from https://www.te.com/usa-en/product-2065657-1.html

\bibitem{2746} NEC UPC2746TB Datasheet , (n.d.), Retrieved August 24, 2018, from http://www.cel.com/pdf/datasheets/up2745tb.pdf

\bibitem{2712} NEC UPC2746TB Datasheet  , (n.d.), Retrieved August 24, 2018, from http://www.cel.com/pdf/datasheets/upc2711.pdf

\bibitem{esd} Electro Static Discharge, (2016, April),
Retrieved August 12, 2018, from https://www.wikizero.pro/index.php?q=aHR0cHM6Ly9lbi53aWtpcGVkaWE\\ub3JnL3dpa2kvRWxlY3Ryb3N0YXRpY19kaXNjaGFyZ2U

\bibitem{osh} Electro Static Discharge, (2016, April),
Retrieved August 12, 2018, from https://www.wikizero.pro/index.php?q=aHR0cHM6Ly9lbi53aWtpcGVkaWE\\ub3JnL3dpa2kvT2NjdXBhdGlvbmFsX3NhZmV0eV9hbmRfaGVhbHRo

\end{thebibliography}

\endgroup

\vfill

%\- \\[5cm]

%\begin{appendices}

%\includepdf[pages=1, scale=0.75 ,pagecommand=\appendices]{up2745tb.pdf}
%\includepdf[pages=2, scale=0.75 ,pagecommand={}]{up2745tb.pdf}
%\includepdf[pages=5-8, scale=0.75 ,pagecommand={}]{up2745tb.pdf}

%\includepdf[pages=1-2, scale=0.75 ,pagecommand={}]{upc2711.pdf}
%\includepdf[pages=5-6, scale=0.75 ,pagecommand={}]{upc2711.pdf}
%\includepdf[pages=8-9, scale=0.75 ,pagecommand={}]{upc2711.pdf}

%\end{appendices}




\tikzset{
desicion/.style={
    diamond,
    draw,
    text width=4em,
    text badly centered,
    inner sep=0pt
},
block/.style={
    rectangle,
    draw,
    text width=10em,
    text centered,
    rounded corners
},
cloud/.style={
    draw,
    ellipse,
    minimum height=2em
},
descr/.style={
    fill=white,
    inner sep=2.5pt
},
connector/.style={
    -latex,
    font=\scriptsize
},
rectangle connector/.style={
    connector,
    to path={(\tikztostart) -- ++(#1,0pt) \tikztonodes |- (\tikztotarget) },
    pos=0.5
},
rectangle connector/.default=-2cm,
straight connector/.style={
    connector,
    to path=--(\tikztotarget) \tikztonodes
}
}

\tikzset{
desicion/.style={
    diamond,
    draw,
    text width=4em,
    text badly centered,
    inner sep=0pt
},
block/.style={
    rectangle,
    draw,
    text width=10em,
    text centered,
    rounded corners
},
cloud/.style={
    draw,
    ellipse,
    minimum height=2em
},
descr/.style={
    fill=white,
    inner sep=2.5pt
},
connector/.style={
    -latex,
    font=\scriptsize
},
rectangle connector/.style={
    connector,
    to path={(\tikztostart) -- ++(#1,0pt) \tikztonodes |- (\tikztotarget) },
    pos=0.5
},
rectangle connector/.default=-2cm,
straight connector/.style={
    connector,
    to path=--(\tikztotarget) \tikztonodes
}
}

\vfill % Fill the rest of the page with whitespace
\end{document}